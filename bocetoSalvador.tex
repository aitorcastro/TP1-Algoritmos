\documentclass[10pt,a4paper]{article}

\input{AEDmacros}
\usepackage{caratula} % Version modificada para usar las macros de algo1 de ~> https://github.com/bcardiff/dc-tex


\titulo{Trabajo Práctico N°1}
\subtitulo{Especificación de TADs}

\fecha{\today}

\materia{Algoritmos y estructura de datos}
\grupo{Grupo Trelew}

\integrante{Apellido, Nombre1}{001/01}{email1@dominio.com}
\integrante{Apellido, Nombre2}{002/01}{email2@dominio.com}
\integrante{Apellido, Nombre3}{003/01}{email3@dominio.com}
\integrante{Apellido, Nombre4}{004/01}{email4@dominio.com}
% Pongan cuantos integrantes quieran

% Declaramos donde van a estar las figuras
% No es obligatorio, pero suele ser comodo
\graphicspath{{../static/}}

\begin{document}



\maketitle

TAD \$Berretacoin  {


\section{Observadores}
obs: cantidadEmitida(none): Z Esto me devuelve cuanta moneda emití hasta ese momento \\

obs: saldo(id: Z) : Z Me da el saldo de un usuario dada una Id\\

obs: transaccion(id_bloque: Z, id_transacción: Z): Transacción . Acá Transacción es una tupla como indica en el enunciado.\\

obs: bloque(id: Z): Bloque . Le doy la id de un bloque y me devuelve la lista de transacciones\\

obs cantidadUsuarios(): Z. Esto lo uso para evitar definirme un tipo de datos usuario \\

\section{Inicializador}

proc inicializarBerretacoin(){

    res.cantidadEmitida() = 0
    res.cantidadUsuarios() = 0


}

\section{Procesos y predicados sobre usuarios: }

\subsection{Procesos}

proc agregarUsuario(inout b: Berretacoin){

    requiere{b = b_0}
    asegura{b.camtidadUsuarios() = b_0.cantidadUsuarios() + 1, y luego b.saldo(b.cantidadUsuarios()) = 0 }
} \\


proc actualizarSaldo(in id: Z, in monto: Z, inout b: Berretacoin ){

    requieque{esUsuarioValido(id) y b = b_0}

    asegura{ transacciob.saldo(id) = b_0.saldo(id)+ monto}\\


\subsection{Predicados}
pred esUsuarioValido(id : Z){
    si y solo si  0 $<$ id $\leq$ cantidadUsuarios()
}\\

pred usuariosDistintos(in id_1: Z, in id_2: Z){

    sii id_1 != id_2
}\\




\section{Procesos y predicacos transacciones}


}\\
pred esDeCreacion(in id_bloque: Z, in id_transaccion: Z){

    res true si y solo si id_bloque $<$ 3000 y luego transaccion(id_bloque, id_transaccion).comprador $==$ 0

}

pred sonTransaccionesValidas(in id_bloque: Z, in b: Berretacoin){
    true sii {long|Bloque| $\leq$ 50 y luego\\
    noHayRepetidos y luego\\
    id_comprador, id_vendedor $\leq$ b_0.cantidadUsuarios() y luego\\
    saldo( quiero poner que el saldo del comprador es mayor o igual que el monto)
    
}

\section{Procesos y predicados pedidos}
\\
    obs blockchain: seq$\langle$Bloque$\rangle$ %Secuancias de bloques ordenados por IDbloque
    \\
    obs saldos: dict$\langle$idUsuario: \ent, dinero: \ent$\rangle$%Diccionario que mapea cada usuario a su saldo
    \\
    obs totalCreado: \ent %Contador de unidades Creadas

\subsection{agregarBloque}

    \begin{proc}{agregarBloque}{\Inout cadena: Berretacoin, \In bloque: Bloque}{}
        \requiere{bloque.id = ultimoBloque(cadena.blockchain).id + 1}
        \requiere{bloqueValido(bloque)} %tiene a lo sumo 50 transacciones ordenadas por idTransaccion y si tiene transaccion de creacion esta en la posicion 0
        \requiere{tieneTransaccionDeCreacion(bloque) \implicaLuego totalCreado < 3000}
        \requiere{\paraTodo[unalinea]{i}{\ent}{
            (0 \leq i < |bloque.transacciones|) \land \lnot esCreativa(bloque.transacciones[i])
            \\ 
            \implicaLuego (montoValido(bloque.transacciones[i]))
            }
        }
        \requiere{cadena = C0}

        \asegura{cadena.blockchain = C0.blockchain + bloque}  
        \asegura{\paraTodo[unalinea]{i}{\ent}{
            (0 \leq i < |bloque.transacciones|) \land esCreativa(bloque.transacciones[i])\\
            \implicaLuego cadena.saldos[getIdVendedor(bloque.transacciones[i])] = \\ C0.saldos[getIdVendedor(bloque.transacciones[i])] + getMonto(bloque.transacciones[i])}
        }
        \asegura{\paraTodo[unalinea]{i}{\ent}{
            (0 \leq i < |bloque.transacciones|) \land esCreativa(bloque.transacciones[i])\\
            \implicaLuego \\
            cadena.saldos[getIdVendedor(bloque.transacciones[i])] = \\
            C0.saldos[getIdVendedor(bloque.transacciones[i])] + getMonto(bloque.transacciones[i])\\
            \land \\
            cadena.saldos[getIdComprador(bloque.transacciones[i])] = \\
            C0.saldos[getIdComprador(bloque.transacciones[i])] - getMonto(bloque.transacciones[i])}
        }
    \end{proc}
\subsection*{auxiliares}
\aux{ultimoBloque}{blockchain: seq$\langle$Bloque$\rangle$}{Bloque}{blockchain[|blockchain| - 1]}
\pred{montoValido}{transaccion: Transaccion, cadena: Berretacoin}
{transaccion.monto \leq cadena.saldos[transaccion.idComprador]}
\aux{getIdVendedor}{transaccion: Transaccion}{\ent}{transaccion.idVendedor}
\aux{getIdComprador}{transaccion: Transaccion}{\ent}{transaccion.idComprador}
\aux{getMonto}{transaccion: Transaccion}{\ent}{transaccion.monto}

1. agregarBloque(inout b: Berretacoin, in bloque: Bloque)
Requisitos:

    - El bloque debe tener un ID mayor que el último bloque en blockchai.

    - Las transacciones dentro del bloque están ordenadas por id_transaccion.

    - Si el bloque contiene una transacción de creación (comprador=0), debe estar en posición 0 y total creado debe ser menor a 3000. (pq si hay mas de 3000 bloques ya no puede haber transacciones de creacion)

    - Todas las transacciones son válidas: saldos suficientes, (creo que lo que sigue se tendria que verificar en TAD transaccion) verificar que id vendedor sea valido y diferente a id compradr y que este tambien sea valido y verificar que el monto sea valido osea positivo.

Postcondiciones:

    - blockchain' = blockchain + [bloque] (agrego el bloque).

    - Para cada transacción en el bloque SOLO SI todas las transacciones son validas:

        - Si es de creación: saldos'[vendedor] = saldos[vendedor] + monto y totalCreado' = totalCreado + 1.

        - Si no: saldos'[comprador] = saldos[comprador] - monto y saldos'[vendedor] = saldos[vendedor] + monto.\\

2. maximosTenedores(out usuarios: seq<Usuario>)
Postcondiciones:

    - lista usuarios contiene todos los usuarios con saldo máximo.

        - creo que algo asi seria la formula: ∀u ∈ usuarios: saldos[u] = max({saldos[v] | v ∈ Usuario}).\\

3. montoMedio(out promedio: R)
Postcondiciones:

    - promedio seria todos los montos de las transacciones de cada bloque con IDcomprador diferente a 0 dividido la cantidad de transacciones con IDcomprador diferente a 0.

    - Si no hay transacciones donde el id comprador es diferente a 0, promedio = 0.

4. cotizacionAPesos(in cotizaciones: seq<Z>, out valores: seq<Z>)
Requisitos:

    - |cotizaciones| = |blockchain|.

Postcondiciones:

    - Para cada bloque i:

    - valores[i] = (total de Berretacoin hasta el bloque i) * cotizaciones[i].

    - El "total hasta el bloque i" incluye todas las creaciones hasta ese bloque.


}


\section{Predicados Generales}
pred noHayRepetidos(in s: seq<T>){

   true sii para todo i != j es s[i] != s[j]
}

\section{Otras ideas}

Hacer todo con 4 observadores:

blockchain = secuendia de Bloques\\

bloque = secuencia de transacciones. donde cada transaccion es un struc \< id_transaccion, id_comprador, id_vendedor, monto \> \\

monedasEmitidas = me da un Z con la cantidad emitida\\

obs usuarios = secuencia de usuario donde usuario es un struc<id, saldo>

Acá debería redefinir las funciones pero es bastante similar. Preguntar si se puede usar esto, ya que haria todo más sencillo.




\section{Preguntas:}

\begin{itemize}
    \item ¿Debo agregar un proceso que sea nuevaTransacción?
    \item ¿El bloque es dado de alguna forma? ¿Viene con id?
    \item ¿puedo usar lo que está en otras ideas?
\end{itemize}

}

\end{document}
